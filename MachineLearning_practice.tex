\documentclass [12pt] {oblivoir}

\usepackage{fapapersize}
\usefapapersize{210mm,297mm,20mm,*,20mm,22mm}

\setlength\parindent{0pt}

\usepackage{helvet}
\renewcommand\familydefault{\sfdefault}
\usepackage[T1]{fontenc}

\usepackage{graphicx}
\usepackage{mathtools}
\usepackage{amsmath}
\usepackage{upgreek}
\usepackage{dsfont}

\usepackage{hyperref}
% \hypersetup{
%     colorlinks=true, %set true if you want colored links
%     linktoc=all,     %set to all if you want both sections and subsections linked
% }

\let\oldsubsubsection=\subsubsection
\renewcommand{\subsubsection}
{
\filbreak
\oldsubsubsection
}

\usepackage{xcolor}
\usepackage{mdframed}

\usepackage{listings}

\title{Machine Learning}
\author{GNU emacser}
\date{}

\setcounter{tocdepth}{3}
\setcounter{secnumdepth}{4}

\begin{document}
\maketitle

\newpage
\tableofcontents

\section {Chapter 05 딥러닝 최적화}

\textbf{\textcolor{gray}{1}} 신경망의 출력이 $(0.4, 2.0, 0.001, 0.32)^{T}$일 때 softmax를 적용한 결과를 쓰시오.

\begin{equation*}
  \begin{pmatrix}
    0.13250 \\
    0.65628\\
    0.08891\\
    0.12231 \\
  \end{pmatrix}
\end{equation*}

\textbf{\textcolor{gray}{2}} softmax를 적용한 후 출력이 $(0.001, 0.9, 0.001, 0.098)^{T}$이고 레이블 정보가 $(0, 0, 0, 1)^{T}$일 때, 세 가지 목적함수 평균 제곱 오차, 교차 엔트로피, 로그우도를 계산하시
오.

\begin{center}
MSE result

\vspace{2mm}
0.811803

\vspace{5mm}
Cross Entropy result

\vspace{2mm}
6.675889369173579

\vspace{5mm}
Log Likelihood result

\vspace{2mm}
3.3510744405468786
\end{center}

\textbf{\textcolor{gray}{3}} [예제 5-1]에서 $\lambda = 0.1, \lambda = 0.5$일 때를 계산하고 $\lambda$에 따른 효과를 설명하시오. 이때 [그림 5-21]을 활용하시오.

\begin{equation*}
  \lambda = 0.25
  \begin{pmatrix}
    1.49159 \\
    1.36074
  \end{pmatrix}
  \lambda = 0.1
  \begin{pmatrix}
    1.61538 \\
    1.27884
  \end{pmatrix}
  \lambda = 0.5
  \begin{pmatrix}
    1.4 \\
    1.4
  \end{pmatrix}
\end{equation*}

\end{document}
